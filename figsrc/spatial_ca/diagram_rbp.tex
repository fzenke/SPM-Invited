\documentclass{standalone}

\usepackage[latin1]{inputenc}
\usepackage{tikz}
\usepackage{tikz,graphicx}
\usepackage{sfmath}
%\usepackage{bbold}
%\usepackage{dsfont}
\usetikzlibrary{shapes,arrows,calc}
\begin{document}
%\pagestyle{empty}

\definecolor{spikecolor}{RGB}{180,51,76}
\definecolor{gradcolor}{RGB}{51,76,180}


% Define block styles
\tikzstyle{block} = [rectangle, draw, text width=3em, text centered, rounded
corners, minimum height=2.2em]
\tikzstyle{sum} = [block, fill=black!10 ]
\tikzstyle{loss} = [block, fill=white ]
\tikzstyle{syn} = [block, fill=black!20 ]
\tikzstyle{mem} = [block, fill=blue!20 ]
\tikzstyle{spk} = [block, fill=spikecolor!20 ]
\tikzstyle{gradedgev} = [draw, thick, gradcolor, transform canvas={xshift=-0.5em}, font=\scriptsize ]
\tikzstyle{randgradedgev} = [draw, dash dot, thick, gradcolor, transform canvas={xshift=-0.5em}, font=\scriptsize ]
\tikzstyle{gradedgeh} = [draw, thick, gradcolor, transform canvas={yshift=-0.5em}, font=\scriptsize ]
\tikzstyle{forwedge} = [draw, thick, black!50, font=\scriptsize]
\tikzstyle{randforwedge} = [draw,dash, thick, black!50, font=\scriptsize]

    
\begin{tikzpicture}[node distance = 1.5cm, auto]
    \pgfmathsetmacro{\n}{0}  % sets number of units -1
    \pgfmathsetmacro{\nm}{-1}  % set to \n -1
    \pgfmathsetmacro{\sep}{2.7}  



	% Place nodes 
	\foreach \t in {0,...,\n} {
    \node [sum] (x1\t) at (\sep*\t,0) {$x^{(0)}$};
		\node [mem, above of=x1\t] (x2\t) {$U^{(1)}$};
		\node [mem, above of=x2\t] (x3\t) {$U^{(2)}$};
    \node [loss, above of=x3\t] (loss\t) {$\mathcal{L}$};
	}


  % Draw vertical edges
	\foreach \t in {0,...,\n} {
    	\path [->, forwedge] (x1\t) -- (x2\t)        ; % node[near start, right] {$W^{(0,1)}$};
    	\path [->, forwedge] (x2\t) -- (x3\t)        ; % node[pos=.5, right] {$W^{(1,2)}$};
    	\path [->, forwedge] (x3\t) -- (loss\t); % node[pos=.5, right] {$W^{(1,2)}$};
		% \path [->, draw, thick, spikecolor] (upstream\t) -- (x\t) node[near end,
		% right] {$W^{(0,1)}$};
	}
  %\path [<-, gradedgev] (x1\n) -- (x2\n)       node[midway, left] {$\delta^1$};
  \path [->, randgradedgev] (loss\n) edge[bend right=35] (x30); % node[pos=.5, left] { $\delta^L$};;
  \path [->, randgradedgev] (loss\n) edge[bend right=35] (x20); % node[pos=.5, left] { $\delta^L$};;
  \path [->, randgradedgev] (loss\n) edge[bend right=35] (x10); % node[pos=.5, left] { $\delta^L$};;



\end{tikzpicture}

\end{document}
